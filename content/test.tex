\section{Mengenlehre}
\subsection*{Teilmenge und Obermenge}
Eine Menge $B$ heißt Teilmenge einer Menge $A$ genau dann,
wenn jedes Element von $B$ auch ein Element von $A$ ist ($B\subseteq A\Leftrightarrow\forall x:x\in B\Rightarrow x\in A$).
$A$ heißt dann Obermenge von $B$. Eine Menge $B$ heißt echte Teilmenge von $A$ ($B\subset A$), falls gilt $B\subseteq A\wedge B\neq A$
Probability density function:
\[
    \begin{cases*}
        \frac{1}{b-a} & for $x\in[a,b]$ \\
        0 & otherwise \\
    \end{cases*}
\]