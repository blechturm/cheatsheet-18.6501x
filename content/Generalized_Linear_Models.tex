\section{OLS}

$Y|X=x \sim N(\mu(x),\sigma^2 I)$\\

Regression function $\mu(x)$:\\

$\mathbb{E}[Y|X=x]=\mu(x) = x^T\beta$\\

Random Component of the Linear Model:

$Y$ is continous and $Y|X=x$ is Gaussian with mean $\mu(x)$


\section{Generalized Linear Models}
We relax the assumption that $\mu$ is linear. Instead, we assume that g $\circ \mu$ is linear, for some function $g$:\\

$g(\mu (\mathbf x)) = \mathbf x^ T \beta$

The function $g$ is assumed to be known, and is referred to as the link function. It maps the domain of the dependent variable to the entire real Line.

it has to be strictly increasing,

it has to be continuously differentiable and

its range is all of $\mathbb{R}$


\subsection{The Exponential Family}

A family of distribution $\, \{ \mathbf{P}_{{\boldsymbol \theta }}: {\boldsymbol \theta }\in \Theta \} ,\,$  where the parameter space $\Theta \subset \mathbb {R}^ k\,$ is -$k$ dimensional, is called a $k$-parameter exponential family on $\mathbb{R}^1$ if the pmf or pdf $\, f_{\boldsymbol \theta }:\mathbb {R}^ q\to \mathbb {R}\,$ of $\, \mathbf{P}_{{\boldsymbol \theta }}\,$ can be written in the form:\\

$\displaystyle  \displaystyle f_{\boldsymbol \theta }(\mathbf{y})=h(\mathbf{y})\, \exp \left({\boldsymbol \eta }({\boldsymbol \theta })\cdot \mathbf{T}(\mathbf{y})-B({\boldsymbol \theta })\right)\qquad \text {where } \\ \begin{cases}  {\boldsymbol \eta }({\boldsymbol \theta })=\begin{pmatrix} \eta _1({\boldsymbol \theta })\\ \vdots \\ \eta _ k({\boldsymbol \theta })\end{pmatrix}& :\mathbb {R}^ k\to \mathbb {R}^ k\\ \mathbf{T}(\mathbf{y})=\begin{pmatrix} T_1(\mathbf{y})\\ \vdots \\ T_ k(\mathbf{y})\end{pmatrix}& :\mathbb {R}^ q\to \mathbb {R}^ k\\ B({\boldsymbol \theta })& :\mathbb {R}^ k\to \mathbb {R}\\ h(\mathbf{y})& :\mathbb {R}^ q\to \mathbb {R}.\\ \end{cases}$\\


if $k=1$ it reduces to:\\

$\displaystyle  \displaystyle f_\theta (y)=h(y)\, \exp \left(\eta (\theta ) T(y)-B(\theta )\right)$

