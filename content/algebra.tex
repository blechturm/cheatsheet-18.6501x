\section{Algebra}
Absolute Value Inequalities:\\
$ |f(x)| < a  \Rightarrow  -a < f(x) < a$\\ 
$|f(x)| > a  \Rightarrow f(x) > a$ or $f(x) < -a$\\
%\section{Matrixalgebra}
\section{Calculus}

Differentiation under the integral sign\\
$\frac{\text{d}}{\text{d}x}\left( \int_{a(x)}^{b(x)}f(x,t)\text{d}t \right ) = f(x,b(x))b'(x)-f(x,a(x))a'(x)+\int_{a(x)}^{b(x)}f_x(x,t)\text{d}t.$

\subsection*{Concavity in 1 dimension}
If $g:I \to \mathbb {R}$ is twice differentiable in the interval $I$:

concave:\\ 
if and only if $g^{\prime \prime }(x) {\color{blue}{\leq }}  0$ for all $x \in I$\\

strictly concave:\\
if $g^{\prime \prime }(x) {\color{blue}{<}}  0$ for all $x \in I$\\

convex:\\ 
if and only if $g^{\prime \prime }(x) {\color{blue}{\geq }}  0$ for all $x \in I$\\

strictly convex if:\\
$g^{\prime \prime }(x) {\color{blue}{>}}  0$ for all $x \in I$\\

\subsection*{Multivariate Calculus}
The Gradient $\nabla$ of a twice differntiable function $f$ is defined as:

$\nabla f:\mathbb {R}^ d \rightarrow \mathbb {R}^ d $\\
$\displaystyle \theta =\begin{pmatrix} \theta _1\\ \theta _2\\ \vdots \\ \theta _ d\end{pmatrix} \displaystyle \mapsto \displaystyle \left.\begin{pmatrix}  \frac{\partial f }{\partial \theta _1}\\ \frac{\partial f }{\partial \theta _2}\\ \vdots \\ \frac{\partial f }{\partial \theta _ d}\end{pmatrix}\right|_{\theta }$\\
\textbf{Hessian}\\

The Hessian of $f$ is a symmetric matrix of second partial derivatives of $f$\\

$\mathbf{H} h(\theta )= \nabla ^{2} h(\theta )= \\
\left(\begin{array}{ c c c }
\frac{\partial ^{2} h}{\partial \theta _{1} \partial \theta _{1}} (\theta ) & \cdots  & \frac{\partial ^{2} h}{\partial \theta _{1} \partial \theta _{d}} (\theta )\\
 & \vdots  & \\
\frac{\partial ^{2} h}{\partial \theta _{d} \partial \theta _{1}} (\theta ) & \cdots  & \frac{\partial ^{2} h}{\partial \theta _{d} \partial \theta _{d}} (\theta )
\end{array}\right) \in \mathbb{R}^{d\times d}$\\

A symmetric (real-valued) $d\times d$  matrix $\mathbf{A}$ is:\\

Positive semi-definite:\\
$\mathbf{x}^ T \, \mathbf{A}\, \mathbf{x} \geq 0\qquad \text {for all }\,  \mathbf{x}\in \mathbb {R}^ d.$\\

Positive definite:\\
$\mathbf{x}^ T \, \mathbf{A}\, \mathbf{x}> 0$ for all non-zero vectors $\mathbf{x}\in \mathbb {R}^ d$\\

Negative semi-definite (resp. negative definite):\\

$\mathbf{x}^ T \, \mathbf{A}\, \mathbf{x}$ is negative for all $\mathbf{x}\in \mathbb {R}^ d-\{ \mathbf{0}\}$.\\

Positive (or negative) definiteness implies positive (or negative) semi-definiteness.\\

If the Hessian is positive definite then $f$ attains a local minimum at $a$ (convex).\\

If the Hessian is negative definite at $a$, then f attains a local maximum at $a$ (concave).\\

If the Hessian has both positive and negative eigenvalues then $a$ is a saddle point for $f$.

%\subsection*{Lagrange Multiplier}