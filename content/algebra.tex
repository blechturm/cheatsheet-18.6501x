\section{Algebra}
Absolute Value Inequalities:\\
$ |f(x)| < a  \Rightarrow  -a < f(x) < a$\\ 
$|f(x)| > a  \Rightarrow f(x) > a$ or $f(x) < -a$\\
\section{Calculus}
\subsection{Concavity in 1 dimension}
If $g:I \to \mathbb {R}$ is twice differentiable in the interval $I$, i.e. $g^{\prime \prime }(x)$ exists for all $x \in I$, then $g$  is

concave if and only if $g^{\prime \prime }(x) {\color{blue}{\leq }}  0$ for all $x \in I$;

strictly concave if $g^{\prime \prime }(x) {\color{blue}{<}}  0$ for all $x \in I$;

convex if and only if $g^{\prime \prime }(x) {\color{blue}{\geq }}  0$ for all $x \in I$;

strictly convex if $g^{\prime \prime }(x) {\color{blue}{>}}  0$ for all $x \in I$;

\subsection{Multivariate Calculus}
\textbf{Gradient}\\
Let\\
$f: \mathbb {R}^ d \longrightarrow \mathbb {R} \theta = \displaystyle \theta =\begin{pmatrix} \theta _1\\ \theta _2\\ \vdots \\ \theta _ d\end{pmatrix} \displaystyle \mapsto \displaystyle f(\theta )$\\
denote a twice differentiable function, the Gradient $\nabla$ of $f$ is defined as:

$\nabla f:\mathbb {R}^ d \rightarrow \mathbb {R}^ d $\\
$\displaystyle \theta =\begin{pmatrix} \theta _1\\ \theta _2\\ \vdots \\ \theta _ d\end{pmatrix} \displaystyle \mapsto \displaystyle \left.\begin{pmatrix}  \frac{\partial f }{\partial \theta _1}\\ \frac{\partial f }{\partial \theta _2}\\ \vdots \\ \frac{\partial f }{\partial \theta _ d}\end{pmatrix}\right|_{\theta }$\\
\textbf{Hessian}\\
The Hessian of $f$ is the matrix $\mathbf{H}: \mathbb {R}^ d \to \mathbb {R}^{d\times d}$ whose entry in the $i$-th row and $j$-th column is defined by\\

$\left(\mathbf{H}\, f\right)_{ij} := \frac{\partial ^2}{\partial \theta _ i \partial \theta _ j} f, \quad 1 \leq i, j \leq d$\\

\textbf{Semi-Definiteness}\\	 	 

A symmetric (real-valued) $d\times d$  matrix $\mathbf{A}$ is:\\

Positive semi-definite if $\mathbf{x}^ T \, \mathbf{A}\, \mathbf{x} \geq 0\qquad \text {for all }\,  \mathbf{x}\in \mathbb {R}^ d.$\\

Positive definite if inequality above is strict $\mathbf{x}^ T \, \mathbf{A}\, \mathbf{x}> 0$ for all non-zero vectors $\mathbf{x}\in \mathbb {R}^ d$\\

Negative semi-definite (resp. negative definite) if $\mathbf{x}^ T \, \mathbf{A}\, \mathbf{x}$ is non-positive (resp. negative) for all $\mathbf{x}\in \mathbb {R}^ d-\{ \mathbf{0}\}$.\\

Positive (or negative) definiteness implies positive (or negative) semi-definiteness.

\textbf{Concavity}\\
	