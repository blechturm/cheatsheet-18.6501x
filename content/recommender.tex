\section{Recommender Systems}

\subsection{K nearest neighbors}

The  $K$-Nearest Neighbor method makes use of ratings by  $K$  other “similar" users when predicting  Yai .

Let  $KNN(a)$  be the set of  K  users “similar to" user  $a$ , and let $sim(a,b)$ be a similarity measure between users  $a$  and  $b \in KNN(a)$. The  K -Nearest Neighbor method predicts a ranking  Yai  to be :

\begin{align*}
\widehat{Y}_{ai} = \displaystyle \frac{\displaystyle \sum _{b \in \text {KNN}(a)} \text {sim}(a,b) Y_{bi}}{\displaystyle \sum _{b \in \text {KNN}(a)} \text {sim}(a,b)}.
\end{align*}

 
The similarity measure  $sim(a,b)$  could be any distance function between the feature vectors  $xa$ and  $x_b$  of users  $a$  and  $b$ , e.g. the euclidean distance  $\left\|  x_ a-x_ b \right\|$  and the cosine similarity  $c\displaystyle \cos \theta = \frac{x_ a\cdot x_ b}{\left\|  x_ a \right\| \left\|  x_ b \right\| }$ . 


\subsection{Collaborative Filtering}

Matrix $Y$ with $n$ rows (users) and $m$ columns (Movies) is sparse (entries missing), $(a,i)$th  entry  $Y_ai$  is the rating by user  $a$  of movie  $i$ if this rating has already been given, and blank if not. Goal is to predict matrix $X$ with no missing entries.

Let $D$  be the set of all $(a,i)$ 's for which a user rating  $Y_ai$  exists, i.e. $(a,i) \in D$ if and only if the rating of user  $a$  to movie  $i$  exists.


\begin{align*}
\displaystyle J &= \sum _{(a,i) \in D} \frac{(Y_{ai} - \big [UV^ T\big ]_{ai})^2}{2} + \frac{\lambda }{2} \left(\sum _{a,k} U_{ak}^2 + \sum _{i,k} V_{ik}^2\right)\\
u&=
\begin{bmatrix}
u_1\\
u_2
\end{bmatrix};  
v=
\begin{bmatrix}
v_1\\
v_2\\
v_3
\end{bmatrix}\\
v&=
\begin{bmatrix}
2\\
7\\
8
\end{bmatrix}\\
uv^T& = \begin{bmatrix}
2u_1 7u_1 8_u1\\
2u_2 7_u1 8_u2
\end{bmatrix}\\
\end{align*}

Take derivative of Objective function $J$ with respect to every user, set it to zero and find respective $u_i$ value:

\begin{align*}
\frac{d}{du_1}( \frac{(7-8u_1)^2}{2} + \frac{\lambda}{2} u^2) = 0\\
\frac{d}{du_2}(J) = 0\\
u_1 = \frac{66}{\lambda + 68};
u_2 = \frac{16}{\lambda +53}
\end{align*}

Use resulting values for $u$ to compute $uv^T$ compare resulting matrix $X$ with matrix $Y$ and start again. Continue until convergence.